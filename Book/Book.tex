\documentclass{book}\usepackage[]{graphicx}\usepackage[]{color}
%% maxwidth is the original width if it is less than linewidth
%% otherwise use linewidth (to make sure the graphics do not exceed the margin)
\makeatletter
\def\maxwidth{ %
  \ifdim\Gin@nat@width>\linewidth
    \linewidth
  \else
    \Gin@nat@width
  \fi
}
\makeatother

\definecolor{fgcolor}{rgb}{0.345, 0.345, 0.345}
\newcommand{\hlnum}[1]{\textcolor[rgb]{0.686,0.059,0.569}{#1}}%
\newcommand{\hlstr}[1]{\textcolor[rgb]{0.192,0.494,0.8}{#1}}%
\newcommand{\hlcom}[1]{\textcolor[rgb]{0.678,0.584,0.686}{\textit{#1}}}%
\newcommand{\hlopt}[1]{\textcolor[rgb]{0,0,0}{#1}}%
\newcommand{\hlstd}[1]{\textcolor[rgb]{0.345,0.345,0.345}{#1}}%
\newcommand{\hlkwa}[1]{\textcolor[rgb]{0.161,0.373,0.58}{\textbf{#1}}}%
\newcommand{\hlkwb}[1]{\textcolor[rgb]{0.69,0.353,0.396}{#1}}%
\newcommand{\hlkwc}[1]{\textcolor[rgb]{0.333,0.667,0.333}{#1}}%
\newcommand{\hlkwd}[1]{\textcolor[rgb]{0.737,0.353,0.396}{\textbf{#1}}}%
\let\hlipl\hlkwb

\usepackage{framed}
\makeatletter
\newenvironment{kframe}{%
 \def\at@end@of@kframe{}%
 \ifinner\ifhmode%
  \def\at@end@of@kframe{\end{minipage}}%
  \begin{minipage}{\columnwidth}%
 \fi\fi%
 \def\FrameCommand##1{\hskip\@totalleftmargin \hskip-\fboxsep
 \colorbox{shadecolor}{##1}\hskip-\fboxsep
     % There is no \\@totalrightmargin, so:
     \hskip-\linewidth \hskip-\@totalleftmargin \hskip\columnwidth}%
 \MakeFramed {\advance\hsize-\width
   \@totalleftmargin\z@ \linewidth\hsize
   \@setminipage}}%
 {\par\unskip\endMakeFramed%
 \at@end@of@kframe}
\makeatother

\definecolor{shadecolor}{rgb}{.97, .97, .97}
\definecolor{messagecolor}{rgb}{0, 0, 0}
\definecolor{warningcolor}{rgb}{1, 0, 1}
\definecolor{errorcolor}{rgb}{1, 0, 0}
\newenvironment{knitrout}{}{} % an empty environment to be redefined in TeX

\usepackage{alltt}

\usepackage{makeidx}
\makeindex
\IfFileExists{upquote.sty}{\usepackage{upquote}}{}
\begin{document}
\title{Stress Managment}
\date{}
\author{Judy Minichelli \and William Bickelmann \and Adrew Mayo}
\maketitle

\tableofcontents


\chapter{Deep Breathing}

https://psychcentral.com/lib/learning-deep-breathing/

Learning Deep Breathing By David Rakal
~ 5 min read
It is thought by many cultures that the process of breathing is the essence of being. A rhythmic process of expansion and contraction, breathing is one example of the consistent polarity we see in nature such as night and day, wake and sleep, seasonal growth and decay and ultimately life and death.

Unlike other bodily functions, the breath is easily used to communicate between these systems, which gives us an excellent tool to help facilitate positive change. It is the only bodily function that we do both voluntarily and involuntarily. We can consciously use breathing to influence the involuntary sympathetic nervous system that regulates blood pressure, heart rate, circulation, digestion and many other bodily functions. Breathing exercises can act as a bridge into those functions of the body of which we generally do not have conscious control.

During times of emotional stress our sympathetic nervous system is stimulated and effects a number of physical responses. Our heart rate rises, we perspire, our muscles tense and our breathing becomes rapid and shallow. If this process happens over a long period of time, the sympathic nervous system becomes over stimulated leading to an imbalance that can effect our physical health resulting in inflammation, high blood pressure and muscle pain to name a few.

Consciously slowing our heart rate, decreasing perspiration and relaxing muscles is more difficult than simply slowing and deepening breathing. The breath can be used to directly influence these stressful changes causing a direct stimulation of the parasympathetic nervous system resulting in relaxation and a reversal of the changes seen with the stimulation of the sympathetic nervous system. We can see how our bodies know to do this naturally when we take a deep breath or sigh when a stress is relieved.

The breathing process can be trained

Breathing can be trained for both positive and negative influences on health. Chronic stress can lead to a restriction of the connective and muscular tissue in the chest resulting in a decrease range of motion of the chest wall. Due to rapid more shallow breathing, the chest does not expand as much as it would with slower deeper breaths and much of the air exchange occurs at the top of the lung tissue towards the head. This results in "chest" breathing. You can see if you are a chest breather by placing your right hand on your chest and your left hand on your abdomen. As you breathe, see which hand rises more. If your right hand rises more, you are a chest breather. If your left hand rises more, you are an abdomen breather.

Chest breathing is inefficient because the greatest amount of blood flow occurs in the lower lobes of the lungs, areas that have limited air expansion in chest breathers. Rapid, shallow, chest breathing results in less oxygen transfer to the blood and subsequent poor delivery of nutrients to the tissues. The good news is that similar to learning to play an instrument or riding a bike, you can train the body to improve its breathing technique. With regular practice you will breathe from the abdomen most of the time, even while asleep.

Note: Using and learning proper breathing techniques is one of the most beneficial things that can be done for both short and long term physical and emotional health.

The benefits of abdominal breathing

Abdominal breathing is also known as diaphragmatic breathing. The diaphragm is a large muscle located between the chest and the abdomen. When it contracts it is forced downward causing the abdomen to expand. This causes a negative pressure within the chest forcing air into the lungs. The negative pressure also pulls blood into the chest improving the venous return to the heart. This leads to improved stamina in both disease and athletic activity. Like blood, the flow of lymph, which is rich in immune cells, is also improved. By expanding the lung's air pockets and improving the flow of blood and lymph, abdominal breathing also helps prevent infection of the lung and other tissues. But most of all it is an excellent tool to stimulate the relaxation response that results in less tension and an overall sense of well being.

Abdominal Breathing Technique\index{Abdominal Breathing Technique}

Breathing exercises such as this one should be done twice a day or whenever you find your mind dwelling on upsetting thoughts or when you are experiencing pain.

Place one hand on your chest and the other on your abdomen. When you take a deep breath in, the hand on the abdomen should rise higher than the one on the chest. This insures that the diaphragm is pulling air into the bases of the lungs.
After exhaling through the mouth, take a slow deep breath in through your nose imagining that you are sucking in all the air in the room and hold it for a count of 7 or as long as you are able, not exceeding 7
Slowly exhale through your mouth for a count of 8. As all the air is released with relaxation, gently contract your abdominal muscles to completely evacuate the remaining air from the lungs. It is important to remember that we deepen respirations not by inhaling more air but through completely exhaling it.
Repeat the cycle four more times for a total of 5 deep breaths and try to breathe at a rate of one breath every 10 seconds or 6 breaths per minute. At this rate our heart rate variability increases which has a positive effect on cardiac health.
Once you feel comfortable with the above technique, you may want to incorporate words that can enhance the exercise. Examples would be to say to yourself the word, relaxation (with inhalation) and stress or anger with exhalation. The idea being to bring in the feeling/emotion you want with inhalation and release those you don't want with exhalation.

In general, exhalation should be twice as long as inhalation. The use of the hands on the chest and abdomen are only needed to help you train your breathing. Once you feel comfortable with your ability to breathe into the abdomen, they are no longer needed.

Abdominal breathing is just one of many breathing exercises. But it is the most important one to learn before exploring other techniques. The more it is practiced, the more natural it will become improving the body's internal rhythm.

Using breathing exercises to increase energy

If practiced over time, the abdominal breathing exercise can result in improved energy throughout the day, but sometimes we are in need of a quick "pick-up." The Bellows breathing exercise also called, the stimulating breath can be used during times of fatigue that may result from driving over distances or when you need to be revitalized at work. It should not be used in place of abdominal breathing but in addition as a tool to increase energy when needed. This breathing exercise is opposite that of abdominal breathing. Short, fast rhythmic breaths are used to increase energy, which are similar to the "chest" breathing we do when under stress. The bellows breath recreates the adrenal stimulation that occurs with stress and results in the release of energizing chemicals such as epinephrine. Like most bodily functions this serves an active purpose, but overuse results in adverse effects as discussed above.
The Bellows Breathing Technique The Stimulating Breath

This yogic technique can be used to help stimulate energy when needed. It is a good thing to use before reaching for a cup of coffee.

Sit in a comfortable up-right position with your spine straight.
With your mouth gently closed, breath in and out of your nose as fast as possible. To give an idea of how this is done, think of someone using a bicycle pump a bellows to quickly pump up a tire. The upstroke is inspiration and the downstroke is exhalation and both are equal in length.
The rate of breathing is rapid with as many as 2-3 cycles of inspiration/expiration per second.
While doing the exercise, you should feel effort at the base of the neck, chest and abdomen. The muscles in these areas will increase in strength the more this technique is practiced. This is truly an exercise.
Do this for no longer than 15 seconds when first starting. With practice, slowly increase the length of the exercise by 5 seconds each time. Do it as long as you are comfortably able, not exceeding one full minute.
There is a risk for hyperventilation that can result in loss of consciousness if this exercise is done too much in the beginning. For this reason, it should be practiced in a safe place such as a bed or chair.
This exercise can be used each morning upon awakening or when needed for an energy boost.


\chapter{Meditation}

http://how-to-meditate.org/how-to-meditate

HOW TO MEDITATE ON LAMRIM
THERE ARE FIVE ESSENTIAL STAGES TO SUCCESSFUL MEDITATION ON LAMRIM:

Preparation
Contemplation
Meditation
Dedication
Subsequent Practice
1. Preparation

If we want to cultivate external crops we begin by making careful preparations. First, we remove from the soil anything that might obstruct their growth, such as stones and weeds. Second, we enrich the soil with compost or fertilizer to give it the strength to sustain growth. Third, we provide warm, moist conditions to enable the seeds to germinate and the plants to grow.

There are three essential preparations for successful meditation:

purifying negativities, accumulating merit, and receiving blessings.
In the same way, to cultivate our inner crops of Dharma realizations we must also begin by making careful preparations. First, we must purify our mind to eliminate the negative karma we have accumulated in the past, because if we do not purify this karma it will obstruct the growth of Dharma realizations. Second, we need to give our mind the strength to support the growth of Dharma realizations by accumulating merit. Third, we need to activate and sustain the growth of Dharma realizations by receiving the blessings of the holy beings.

Prayers For Meditation

If you like, you can engage in these preparatory practices by reciting the following prayers while contemplating their meaning,

Going for refuge

(We imagine ourself and all other living beings going for refuge
while reciting three times):

I and all sentient beings, until we achieve enlightenment,
Go for refuge to Buddha, Dharma, and Sangha. (3x, 7x, 100x, or more)

Generating bodhichitta

Through the virtues I collect by giving and other perfections,
May I become a Buddha for the benefit of all. (3x)

Generating the four immeasurables

May everyone be happy,
May everyone be free from misery,
May no one ever be separated from their happiness,
May everyone have equanimity, free from hatred and attachment.

Visualizing the Field for Accumulating Merit

In the space before me is the living Buddha Shakyamuni surrounded
by all the Buddhas and Bodhisattvas, like the full moon surrounded by stars.

Prayer of seven limbs

With my body, speech, and mind, humbly I prostrate,
And make offerings both set out and imagined.
I confess my wrong deeds from all time,
And rejoice in the virtues of all.
Please stay until samsara ceases,
And turn the Wheel of Dharma for us.
I dedicate all virtues to great enlightenment.

Offering the mandala

The ground sprinkled with perfume and spread with flowers,
The Great Mountain, four lands, sun and moon,
Seen as a Buddha Land and offered thus,
May all beings enjoy such Pure Lands.

I offer without any sense of loss
The objects that give rise to my attachment, hatred, and confusion,
My friends, enemies, and strangers, our bodies and enjoyments;
Please accept these and bless me to be released directly from the three poisons.
IDAM GURU RATNA MANDALAKAM NIRYATAYAMI

Prayer of the Stages of the Path

The path begins with strong reliance
On my kind Teacher, source of all good;
O Bless me with this understanding
To follow him with great devotion.

This human life with all its freedoms,
Extremely rare, with so much meaning;
O Bless me with this understanding
All day and night to seize its essence.

My body, like a water bubble,
Decays and dies so very quickly;
After death come results of karma,
Just like the shadow of a body.

With this firm knowledge and remembrance
Bless me to be extremely cautious,
Always avoiding harmful actions
And gathering abundant virtue.

Samsara's pleasures are deceptive,
Give no contentment, only torment;
So please bless me to strive sincerely
To gain the bliss of perfect freedom.

O Bless me so that from this pure thought
Come mindfulness and greatest caution,
To keep as my essential practice
The doctrine's root, the Pratimoksha.

Just like myself all my kind mothers
Are drowning in samsara's ocean;
O So that I may soon release them,
Bless me to train in bodhichitta.

But I cannot become a Buddha
By this alone without three ethics;
So bless me with the strength to practise
The Bodhisattva's ordination.

By pacifying my distractions
And analyzing perfect meanings,
Bless me to quickly gain the union
Of special insight and quiescence.

When I become a pure container
Through common paths, bless me to enter
The essence practice of good fortune,
The supreme vehicle, Vajrayana.

The two attainments both depend on
My sacred vows and my commitments;
Bless me to understand this clearly
And keep them at the cost of my life.

By constant practice in four sessions,
The way explained by holy Teachers,
O Bless me to gain both the stages,
Which are the essence of the Tantras.

May those who guide me on the good path,
And my companions all have long lives;
Bless me to pacify completely
All obstacles, outer and inner.

May I always find perfect Teachers,
And take delight in holy Dharma,
Accomplish all grounds and paths swiftly,
And gain the state of Vajradhara.

Receiving blessings and purifying

From the hearts of all the holy beings,
streams of light and nectar flow down,
granting blessings and purifying.

(At this point we begin the actual contemplation and meditation. After the meditation we dedicate our merit while reciting the following prayers:)

Dedication prayers

Through the virtues I have collected
By practising the stages of the path,
May all living beings find the opportunity
To practise in the same way.

May everyone experience
The happiness of humans and gods,
And quickly attain enlightenment,
So that samsara is finally extinguished.

2. Contemplation

The purpose of contemplation is to bring to mind the object of placement meditation. We do this by considering various lines of reasoning, contemplating analogies, and reflecting on the scriptures. It is helpful to memorize the contemplations given in each section so that we can meditate without having to look at the text. The contemplations given here are intended only as guidelines. We should supplement and enrich them with whatever reasons and examples we find helpful.

By training in Lamrim meditation eventually we shall be able to maintain

a peaceful mind continuously, throughout our life.
3. Meditation

When through our contemplations the object appears clearly, we leave our analytical meditation and concentrate on the object single-pointedly. This single-pointed concentration is the third part, the actual meditation.

When we first start to meditate, our concentration is poor; we are easily distracted and often lose our object of meditation. Therefore, to begin with we shall probably need to alternate between contemplation and placement meditation many times in each session.

For example, if we are meditating on compassion we begin by contemplating the various sufferings experienced by living beings until a strong feeling of compassion arises in our heart. When this feeling arises we meditate on it single-pointedly. If the feeling fades, or if our mind wanders to another object, we should return to analytical meditation to bring the feeling back to mind. When the feeling has been restored we once again leave our analytical meditation and hold the feeling with single-pointed concentration.

More detailed instructions on the contemplations and on Lamrim meditation in general can be found in Modern Buddhism and Joyful Path of Good Fortune.

4. Dedication

Dedication directs the merit produced by our meditation towards the attainment of Buddhahood. If merit is not dedicated it can easily be destroyed by anger. By reciting the dedication prayers sincerely at the end of each meditation session we ensure that the merit we created by meditating is not wasted but acts as a cause for enlightenment.

5. Subsequent Practice

This consists of advice on how to integrate Lamrim meditation into our daily life. It is important to remember that Dharma practice is not confined to our activities during the meditation session; it should permeate our whole life. We should not allow a gulf to develop between our meditation and our daily life, because the success of our meditation depends upon the purity of our conduct outside the meditation session. We should keep a watch over our mind at all times by applying mindfulness, alertness, and conscientiousness; and we should try to abandon whatever bad habits we may have.

Deep experience of Dharma is the result of practical training over a long period of time, both in and out of meditation, therefore we should practise steadily and gently, without being in a hurry to see results.

Summary of meditation on Lamrim

To summarize, our mind is like a field. Engaging in the preparatory practices is like preparing the field by removing obstacles caused by past negative actions, making it fertile with merit, and watering it with the blessings of the holy beings. Contemplation and meditation are like sowing good seeds, and dedication and subsequent practice are the methods for ripening our harvest of Dharma realizations.


\chapter{Chocolate}

http://yalescientific.org/thescope/2017/02/the-science-behind-chocolate/

The Science Behind Chocolate
We love chocolate. The way it melts, the luxurious, silky texture-the way it leaves us wanting more. We celebrate holidays surrounded by the exchange of chocolate as a symbol of love. We read books for a chance to explore chocolate factories and drink from chocolate fountains. And, in perhaps one of the most iconic lines from the movie Forrest Gump, we have held ourselves to the advice "Life is like a box of chocolate. You never know what you're gonna get." However, for chocolatiers, this statement may suggest more than chocolate variety-at the microscopic level, there is more than meets the eye when it comes to the chocolate we consume.

Chocolate with a temper
Chocolate is considered a polymorph, which means it can take on different shapes when it solidifies from liquid form. It has six (types I-VI) different ways it may crystallize after it has been melted and cooled, varying by temperature. The fat composition in cocoa butter-composed of oleic, palmitic, and stearic fatty acids-found in chocolate is largely responsible for these crystal structures. As chocolate is heated and cooled, a process called tempering, chocolatiers are able to manipulate the crystal structure of solid chocolate, creating its desirable form. When chocolate has been improperly stored or tempered, the cocoa butter separates from the chocolate and surfaces, creating whiteness on the chocolate called "chocolate bloom" (but don't worry-it's still safe to eat). Considering how particular the crystal structure dictates the texture of chocolate, it is no surprise that chocolatiers must work with precision to produce the glossy, firm, chocolate that melts in your mouth.

Giving you the meltdown 
Chocolatiers melt down chocolate to change its flavor and shape. However, the temperature matters: heating chocolate only to around 63 degrees Fahrenheit (17 degrees Celsius) before cooling it results in a soft, crumbly product that melts easily. This is characteristic of type I chocolate. However, the sweet spot chocolatiers aim for is around 93 degrees Fahrenheit (34 degrees Celsius). As the chocolate is heated, the crystal structure of the chocolate melts away. Reaching the desired temperature, the chocolate is then cooled, and as it cools, it forms crystals characteristic of type IV and V. However, they then reheat the chocolate to get rid of any type IV crystals, leaving only the desired type V chocolate. Tempering chocolate to this temperature is hitting jackpot-the chocolate, once solidified, is firm, snaps easily and melts at body temperature. As it melts in your mouth at this temperature having been properly tempered with precision, the chocolate disperses, releasing its flavor. It's hard not to love chocolate. But what makes us like chocolate so much?

Why are we so in love with chocolate? 
Every time you consume chocolate, you may have noticed a feeling of satisfaction and a desire to eat more. While your heart feels content, it is actually your brain at work. Inside chocolate is phenylethylamine, a compound that stimulates the brain to release a neurotransmitter called dopamine. Dopamine helps control our body's reward-system, and is released anytime we have a positive experience-from laughing, to eating, to having sex. When we eat chocolate, dopamine is released into the frontal lobe, hippocampus, and hypothalamus, which in turn help regulate our emotions and association of satisfaction and happiness with simply the experience of chocolate consumption. As we associate chocolate with our positive emotions, we desire more chocolate because of its link to happiness. In this regard, it is not surprising why chocolate is a symbol of love.

Not for everyone-why dogs can't eat chocolate 
There's an alkaloid called theobromine in chocolate which functions similarly to caffeine, though at a lower extent. The darker the chocolate, the higher the amount of theobromine. While its stimulating properties have been used for medicinal applications, such as blood vessel widening or heart stimulating, animals such as dogs feel a more drastic effect from this alkaloid. Consumption of theobromine is toxic to dogs; because dogs metabolize this alkaloid much faster than humans, they face more serious side effects, from minor diarrhea to heart attacks.

Learning the structure of chocolate responsible for the sensation we experience as it melts in our mouth, as well as its effect on the brain, enables us to appreciate chocolate more. Though the process of crafting chocolate requires the skill of someone experienced in manipulating its crystal structure, it is consumed and enjoyed by everyone. And just as those in television commercials who are taken away as they eat a piece of chocolate, we experience a special feeling as we unwrap and bite into a piece-it is a moment of bliss.

\printindex

\end{document}
